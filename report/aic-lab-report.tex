\documentclass[parskip,oneside,colorbacktitle,10pt,accentcolor=tud1b]{tudreport}


%% Spracheinstellungen
\usepackage[ngerman,german]{babel}
\usepackage[utf8]{inputenc}
\usepackage[T1]{fontenc} 
\usepackage{microtype} % optischer Randausgleich bei pdflatex mit Zeichendehnung
\usepackage{afterpage} 

%% Grafikeinstellungen
\usepackage{float} % u.a. genaue Plazierung von Gleitobjekten mit H
\usepackage{wrapfig}

%% Tabelleneinstellungen
\usepackage{booktabs}
\usepackage{multirow}
\usepackage{longtable}
\usepackage{tabularx}

%% Mathematik
\usepackage{amsmath}
\usepackage{nicefrac}
\usepackage{icomma}

%% sonstige Einstellungen
\usepackage{paralist}% erweiterte Listenumgebung (z.B. compactitem)
\usepackage{textcomp} % verschiedene Symbole
\usepackage[nottoc, numbib]{tocbibind}
\usepackage{hyperref}
\renewcommand\plparsep{1ex}
\usepackage{enumerate}

\raggedbottom


%% Output Matrikelnummer?
\newcommand{\myAuthor}{Summer Term 2014\\Group 02 (Harshad Dhotre, Peter Schuster)}


\title{Advanced Integrated Circuit Design Lab}
\subtitle{3-bit Resistor String DAC}
\subsubtitle{\myAuthor}
\begin{document}

%% Titel %%%%%%%%%%%%%%%%%%%%%%%%%%%%%%%%%%%%%%%%%%%%%%%%%%%%%%%%%%%%%%%%%%
\maketitle

\setlength{\parindent}{0pt}
\setlength{\parskip}{5pt}

\pagestyle{myheadings}
%\pagenumbering{arabic}

\mymarkright{\myAuthor}

\chapter{Introduction}

\chapter{Design}

\section{OpAmp}

\subsection{Hand Calculations}

\begin{equation}
S_i = \frac{W_i}{L_i}
\end{equation}

\begin{equation}
C_c = 0.5 \cdot C_L
\end{equation}

\begin{equation}
I_5 = C_c \cdot \text{SR}
\end{equation}

\begin{equation}
S_3 = S_4 = \frac{I_5}{{K^{'}_p} \cdot  ( V_{DD} - V_{\text{in(max)}} - | V_{T0p,max} | + V_{T0n,min}  )^2 }
\end{equation}

\begin{equation}
g_{m1} = g_{m2} = \text{GB} \cdot C_c
\end{equation}

\begin{equation}
S_1 = S_2 = \frac{g_{m1}^2}{K^{'}_n \cdot I_5}
\end{equation}

\begin{equation}
V_{ds5,sat} = 0.20 V
\end{equation}

\begin{equation}
S_5 = \frac{2 \cdot I_5}{K^{'}_n \cdot V_{ds5,sat}^2}
\end{equation}

\begin{equation}
g_{m6} = 5 \cdot g_{m2} \cdot \frac{C_L}{C_c}
\end{equation}

\begin{equation}
I_3 = 0.5 \cdot I_5
\end{equation}

\begin{equation}
g_{m3} = g_{m4} = \sqrt{2 \cdot K^{'}_{p} \cdot S_3 \cdot I_3}
\end{equation}

\begin{equation}
S_6 = \frac{g_{m6}}{g_{m4}} \cdot S_4
\end{equation}

\begin{equation}
I_6 = \frac{g_{m6}^2}{2 \cdot K^{'}_{p} \cdot S_6}
\end{equation}

\begin{equation}
S_7 = \frac{I_6}{I_5} \cdot S_5
\end{equation}

\subsection{Reference Voltage}

\subsection{Simulations}

\section{Control Logic}

\section{Analog Multiplexer}

\section{Resistor String}

\chapter{Conclusion}

\end{document}
